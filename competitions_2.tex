\documentclass[11pt, a4paper]{template}

\begin{document}

\begin{titlepage}
  \begin{center}
	\Huge{Задачи всероссийской олимпиады школьников по математике
}
  \end{center}
\end{titlepage}

\chapter{1992-1993}

\begin{exercise}
Натуральное число $n$ таково, что числа $2n + 1$ и $3n + 1$ являются
квадратами. Может ли при этом число $5n + 3$ быть простым?
\end{exercise}

\begin{solution}

\end{solution}

\begin{exercise}
В колоде $n$ карт. Часть из них лежит рубашками вверх, остальные - рубашками вниз. За один ход разрешается взять несколько карт сверху, перевернуть полученную стопку и снова положить ее сверху колоды. За какое наименьшее число ходов при любом начальном расположении карт можно добиться того, чтобы все карты лежали рубашками вниз?
\end{exercise}

\chapter{1993-1994}

\begin{exercise}
На столе лежат три кучки спичек. В первой кучке находится 100 спичек, во второй - 200, а в третьей - 300. Двое играют в такую игру. Ходят по очереди, за один ход игрок должен убрать одну из кучек, а любую из оставшихся разделить на две непустые части. Проигравшим считается тот, кто не может сделать ход. Кто выигрывает при правильной игре: начинающий или его партнер?
\end{exercise}

\begin{exercise}
В правильном $(6n+1)$-угольнике $K$ вершин покрашено в красный цвет, а остальные - в синий. Докажите, что количество равнобедренных треугольников с одноцветными вершинами не зависит от способа раскраски.
\end{exercise}

\begin{solution}
Введем обозначение $N = 6n + 1$. Заметим, что среди треугольников нет равносторонних и отрезок между любыми двумя вершинами принадлежит ровно трем равнобедренным треугольникам. Составим соотношения между количествами треугольников и отрезков. $T$ - количества треугольников, $L$  количества отрезков.
\begin{itemize}
\item Количество синих отрезков, в силу того, что каждый такой отрезок входит либо в синий треугольник (который содержит три таких отрезка), либо в треугольник с двумя синими вершинами (содержит один такой отрезок) - $3L_{b} = 3 T_{bbb} + T_{bb}$.
\item Количество красных отрезков вычисляется аналогично - $3L_{r} = 3T_{rrr} + T_{rr}$.
\item Количество красно-синих отрезков, так как их два в треугольнике с двумя синими вершинами и два в треугольнике с двумя красными вершинами - $3L_{rb} = 2T_{bb}+2T_{rr}$
\end{itemize}
Тогда количество одноцветных треугольников
$$
T_{rrr} + T_{bbb} = \frac{(3L_{b} - T_{bb}) + (3L_{r} - T_{rr})}{3} = L_{b} + L{r} - \frac{L_{rb}}{2} = \binom{N - K}{2} + \binom{K}{2} - \frac{K(N - K)}{2}
$$
\end{solution}

\chapter{2014-2015}

\begin{exercise}
Назовём натуральное число почти квадратом, если оно равно произведению двух последовательных натуральных чисел. Докажите, что каждый почти квадрат можно представить в виде частного двух почти квадратов.
\end{exercise}


\end{document} 
