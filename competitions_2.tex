\documentclass[11pt, a4paper]{template}

\begin{document}

\begin{titlepage}
  \begin{center}
	\Huge{Задачи всероссийской олимпиады школьников по математике
}
  \end{center}
\end{titlepage}

\chapter{1992-1993}

\begin{exercise}
Натуральное число $n$ таково, что числа $2n + 1$ и $3n + 1$ являются
квадратами. Может ли при этом число $5n + 3$ быть простым?
\end{exercise}

\begin{solution}

\end{solution}

\begin{exercise}
Найдите наибольшее натуральное число, из которого вычеркиванием цифр нельзя получить число, делящееся на 11.
\end{exercise}

\begin{solution}
Очевидно, что в числе не должно быть нуля или двух одинаковых цифр. Исходя из этого, рассмотрим в качестве кандидата число 987654321. 
\end{solution}


\begin{exercise}
Докажите, что уравнение $x^{3} + y^{3} = 4(x^{2} y + xy^{2} + 1)$ не имеет решений в целых числах.
\end{exercise}

\begin{solution}
Перепишем уравнение в форме
$$
(x + y)^{3} - 7xy(x + y) = 4
$$
Теперь заметим, что левая часть по модулю 7 не может быть равной 4. 
\end{solution}

\begin{exercise}
Найдите все натуральные числа $n$, для которых сумма цифр числа $5^{n}$ равна $2^{n}$.
\end{exercise}

\begin{solution}
Для $n < 6$ находим единственное решение $n = 3$. Докажем что для $n \geqslant 6$
решений нет. Сумма цифр числа $5^{n}$ не превосходит $9n$. Для $n \geqslant 6$ выполняется $2^{n} > 9n$.
\end{solution}

\begin{exercise}
В колоде $n$ карт. Часть из них лежит рубашками вверх, остальные - рубашками вниз. За один ход разрешается взять несколько карт сверху, перевернуть полученную стопку и снова положить ее сверху колоды. За какое наименьшее число ходов при любом начальном расположении карт можно добиться того, чтобы все карты лежали рубашками вниз?
\end{exercise}

\chapter{1993-1994}

\begin{exercise}
На столе лежат три кучки спичек. В первой кучке находится 100 спичек, во второй - 200, а в третьей - 300. Двое играют в такую игру. Ходят по очереди, за один ход игрок должен убрать одну из кучек, а любую из оставшихся разделить на две непустые части. Проигравшим считается тот, кто не может сделать ход. Кто выигрывает при правильной игре: начинающий или его партнер?
\end{exercise}

\begin{exercise}
В правильном $(6n+1)$-угольнике $K$ вершин покрашено в красный цвет, а остальные - в синий. Докажите, что количество равнобедренных треугольников с одноцветными вершинами не зависит от способа раскраски.
\end{exercise}

\begin{solution}
Введем обозначение $N = 6n + 1$. Заметим, что среди треугольников нет равносторонних и отрезок между любыми двумя вершинами принадлежит ровно трем равнобедренным треугольникам. Составим соотношения между количествами треугольников и отрезков. $T$ - количества треугольников, $L$  количества отрезков.
\begin{itemize}
\item Количество синих отрезков, в силу того, что каждый такой отрезок входит либо в синий треугольник (который содержит три таких отрезка), либо в треугольник с двумя синими вершинами (содержит один такой отрезок) - $3L_{b} = 3 T_{bbb} + T_{bb}$.
\item Количество красных отрезков вычисляется аналогично - $3L_{r} = 3T_{rrr} + T_{rr}$.
\item Количество красно-синих отрезков, так как их два в треугольнике с двумя синими вершинами и два в треугольнике с двумя красными вершинами - $3L_{rb} = 2T_{bb}+2T_{rr}$
\end{itemize}
Тогда количество одноцветных треугольников
$$
T_{rrr} + T_{bbb} = \frac{(3L_{b} - T_{bb}) + (3L_{r} - T_{rr})}{3} = L_{b} + L{r} - \frac{L_{rb}}{2} = \binom{N - K}{2} + \binom{K}{2} - \frac{K(N - K)}{2}
$$
\end{solution}

\chapter{1994-1995}

\begin{exercise}
Можно ли расставить по кругу 1995 различных натуральных чисел так, чтобы для любых двух соседних чисел отношение большего из них к меньшему было простым числом?
\end{exercise}

\begin{solution}
Обозначим эти числа как $a_{1}, a_{2}, a_{3}, \dots, a_{1995}$ и рассмотрим произведение
$$
\frac{a_{2}}{a_{1}} \frac{a_{3}}{a_{2}}\frac{a_{4}}{a_{3}}\cdots \frac{a_{1}}{a_{1995}}
$$
С одной стороны оно равно единице. А с другой стороны
$$
\frac{p_{1} p_{2} \dots p_{n}}{q_{1}q_{2}\dots q_{m}}
$$
Где и $q_{i}$ и $p_{i}$ простые, а также $n + m = 1995$. Но тогда получается что $n$ и $m$ разной четности. А значит эта дробь никак не может быть равной 1. И расставить числа нельзя.
\end{solution}

\begin{exercise}
Все стороны и диагонали правильного 12-угольника раскрашиваются в 12 цветов (каждый отрезок - одним цветом). Существует ли такая раскраска, что для любых трех цветов найдутся три вершины, попарно соединенные между собой отрезками этих цветов?
\end{exercise}

\begin{solution}
Рассмотрим треугольники где одно ребро определенного цвета. Например синего. Если две вершины соединяет ребро синего цвета, то это ребро покрывает 10 сочетаний цветов двух других ребер. Всего же сочетаний которые необходимо покрыть $\binom{11}{2} = 55$. Значит ребро синего цвета согласно принципу Дирихле встречается не менее $\lceil \frac{55}{10} \rceil = 6$ раз. Но это должно выполняться для любого цвета. То есть всего должно быть не менее $6 \cdot 12 = 72$ ребер. Но в графе всего $\binom{12}{2} = 66$ ребер. 
\end{solution}

\begin{exercise}
Натуральные числа $m$ и $n$ таковы, что
$$
\lcm(m, n) + \gcd(m, n) = m + n 
$$
Докажите, что одно из чисел $m$ или $n$ делится на другое.
\end{exercise}

\begin{solution}
Сделаем замену $m = p \cdot \gcd(m, n)$ и $n = q \cdot \gcd(m, n)$. 
$$
\lcm(m, n) + \gcd(m, n) = pq \cdot \gcd(m, n) + gcd(m, n) = (p + q) \cdot \gcd(m, n)
$$
Откуда получаем, что
$$
pq + 1 = p + q \implies (p - 1)(q - 1) = 0
$$
Откуда следует утверждение задачи.
\end{solution}

\chapter{1998-1999}

\chapter{1999-2000}

\begin{exercise}
На доску последовательно выписываются числа $a_{1} = 1, a_{2}, a_{3}, \dots$ по следующим правилам: $a_{n + 1} = a_{n} - 2$, если число $a_{n} - 2$ - натуральное и еще не выписано на доску, в противном случае $a_{n + 1} = a_{n} + 3$. Докажите, что все квадраты натуральных чисел появятся в этой последовательности при прибавлении 3 к предыдущему числу.
\end{exercise}

\begin{solution}
Выпишем первые члены этой последовательности
$$
1, 4, 2, 5, 3, 6, 9, 7, 10, 8, 11, 14, 12, 15, 13, 16, 19, 17, 20, 18, 21, 24, 22, 25, \dots
$$
Теперь по индукции докажем, что если $5 | n$, то на доске уже выписана перестановка чисел от 1 до $n$. База приведена выше. Теперь допустим, что на доске уже выписана перестановка чисел от 1 до $n$ и последнее число в ней $n - 2$. Тогда следующие числа, которые будут выписаны
$$
n + 1, n + 4, n + 2, n + 5, n + 3
$$
Известно, что квадрат числа может давать при делении на 5 только три остатка - 0, 1, 4. Но в приведенной выше последовательности эти остатки могут давать только элементы полученные прибавлением тройки к предыдущему элементу.
\end{solution}

\chapter{2006-2007}

\begin{exercise}
Среди натуральных чисел от 1 до 1200 выбрали 372 различных числа так, что никакие два из них не различаются на 4, 5 или 9. Докажите, что число 600 является одним из выбранных.
\end{exercise}

\begin{solution}
Докажем следующую лемму - из 13 подряд идущих чисел можно выбрать не более 4 так, чтобы никакие из них не различались на 4, 5 или 9. Разобьем числа от $a$ до $a + 12$ на группы.
$$
(a, a + 4, a + 9), (a + 1, a + 5, a + 10), (a + 2, a + 6, a + 11), (a + 3, a + 7, a + 12), (a + 8) 
$$
Заметим, что из каждой группы можно выбрать не более одного элемента. Допустим выбрано по элементу из первых 4 групп и единственный элемент последней группы. Но  этот элемент может быть только с центральным элементом четвертой группы. Продолжая это рассуждение получим группу $(a + 8, a + 7, a + 6, a + 5, a + 4)$. Первый и последний элементы в ней не могут быть вместе. Лемма доказана. Теперь исключим числа 599, 600, 601, 602. Оставшиеся числа составят $\frac{598}{13} + \frac{1200 - 603 + 1}{13} = 92$ группы по 13 чисел.  В них, исходя из леммы, может быть выбрано $92 \cdot 12 = 368$ чисел. Значит числа 599, 600, 601, 602 тоже выбраны. 
\end{solution}

\chapter{2014-2015}

\begin{exercise}
Назовём натуральное число почти квадратом, если оно равно произведению двух последовательных натуральных чисел. Докажите, что каждый почти квадрат можно представить в виде частного двух почти квадратов.
\end{exercise}


\end{document} 
