\documentclass[11pt, a4paper]{template}

\begin{document}

\begin{titlepage}
  \begin{center}
	\Huge{Математические задачи, задачки и головоломки}
  \end{center}
\end{titlepage}

\chapter{2024}

\begin{exercise}
Докажите, что существует бесконечно много натуральных $n$, для которых десятичная запись числа $3n^2$ начинается с $n$ (всех цифр $n$ в десятичной записи, в том же порядке).
\end{exercise}

\begin{solution}
Запишем условие в виде
$$
3n^{2} = 10^{m} n + b
$$
где $b < 10^{m}$. Рассмотрим число вида 
$$
\overline{33\dots 34} = \frac{10^{m} + 2}{3} 
$$
Как видно из
$$
3n^{2} - 10^{m}n = \frac{10^{2m} + 4 \cdot 10^{m} + 4}{3} - \frac{10^{2m} + 2 \cdot 10^{m}}{3} = \frac{2}{3}(10^{m} + 2) < 10^{m}
$$
это является решением задачи. 
\end{solution}

\begin{exercise}
Доказать, что существует бесконечно много троек попарно взаимно-простых натуральных чисел $a$, $b$, $c$ таких, что $a^{2} + b^{3} = c^{4}$.
\end{exercise}


\chapter{2021}

\begin{exercise}
$\text{Reverse}(n)$ - "обращение" десятичной записи числа $n$, то есть число, записанное теми же цифрами в обратном порядке. Докажите, что уравнение $\text{Reverse}(n) = \frac{4n}{7}$ имеет бесконечно много решений.
\end{exercise}

\begin{solution}
Пусть $n = 10a + b$. Тогда $\text{Reverse}(n) = 10^{m} b + a$. И $a < 10^{m}$. Запишем уравнение
$$
\frac{4n}{7} = \text{Reverse}(n) \implies 40a + 4b = 7(10^{m}b + a) \implies 33a = b(7 \cdot 10^{m} - 4)
$$
Заметим, что $7 \cdot 10^{m} - 4$ при нечетном $m$ будет делиться нацело на 33. И, например, при $m = 1$ получим $a = 2b$. Положив $b = 1$ получим $n = 21$. Легко проверить, что оно удовлетворяет условию задачи. Заметим, что периодическое решение $\overline{21}$ тоже будет решением уравнения. И оно тоже будет генерироваться формулой
$$
a = \frac{7 \cdot 10^{2p + 1} - 4}{33} b
$$
\end{solution}

\chapter{2020}

\begin{exercise}
Сколькими способами можно расставить по кругу 21 цифру, используя только цифры от 1 до 7, чтобы каждая пара различных цифр где-то на круге была соседями?
\end{exercise}

\begin{exercise}
Рассмотрим все упорядоченные четверки натуральных чисел $(a,b,c,d)$ с суммой 239. Докажите, что сумма \\ произведений $abcd$, взятая по всем таким четверкам, кратна 239.
\end{exercise}

\chapter{2016}

\begin{exercise}
В салоне самолета $N$ мест. $N$ пассажиров садятся на места по очереди следующим образом:
\begin{enumerate}
\item Первый пассажир садится на произвольное место
\item Каждый следующий пассажир садится на свое место, если оно свободно или на любое, если его место занято.
\end{enumerate}
Какова вероятность того, что последний пассажир сядет на свое место?
\end{exercise}

\end{document} 
