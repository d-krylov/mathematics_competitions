\documentclass[11pt, a4paper]{template}

\begin{document}

\begin{titlepage}
  \begin{center}
	\Huge{Задачи Международного Математического Турнира Городов}
  \end{center}
\end{titlepage}


\chapter{1980-1981}

\begin{exercise}
На бесконечной плоскости играют двое: один передвигает одну фишку-волка, другой - одну из $K$ фишек-овец. После хода волка ходит какая-нибудь из овец, затем, после следующего хода волка, опять какая-нибудь из овец и так далее. И волк, и овцы передвигаются за один ход в любую сторону не более чем на один метр. Верно ли, что для любого числа овец, участвующих в игре, существует такая первоначальная позиция, что волк не поймает ни одной овцы?
\end{exercise}


\chapter{1981-1982}

\begin{exercise}
Квадрат разбит на $K^{2}$ равных квадратиков. Про некоторую ломаную известно, что она проходит через центры всех квадратиков (ломаная может пересекать сама себя). Каково минимальноечисло звеньев у этой ломаной?
\end{exercise}

\begin{exercise}
Рассматривается последовательность
$$
1, \frac{1}{2}, \frac{1}{3}, \frac{1}{4}, \frac{1}{5}, \frac{1}{6}, \frac{1}{7}, \dots
$$
Существует ли арифметическая прогрессия
\begin{itemize}
\item длины 5
\item сколь угодно большой длины,составленная из членов этой последовательности?
\end{itemize}
\end{exercise}
 
\begin{solution}
Существует для обоих пунктов. Возьмем дробь вида $\frac{1}{n!}$. Любая кратная ей дробь, имеющая вид
$\frac{k}{n!}$ будет членом последовательности если $k \leqslant n$. Таким образом получим арифметическую прогрессию длины $k$ с шагом $\frac{1}{n!}$.
\end{solution}

\chapter{1982-1983}

\begin{exercise}
Числа от 1 до 1000 расставлены по окружности. Докажите, что их можно соединить 500 непересекающимися отрезками, разность чисел на концах которых (по модулю) не более 749.
\end{exercise}

\begin{solution}
Распределим числа в два множества - от 251 до 750 в первом и все оставшиеся во втором. Пусть одно множество будет покрашено в зеленый цвет, а другое в синий. Заметим что мощности этих множеств равны и что разность любых двух чисел разного цвета удовлетворяет условию задачи. Очевидно, что найдутся два соседних числа разного цвета. Соединим их прямой. Выкинем эту пару из окружности. Действуя по аналогии соединим все числа непересекающимися отрезками. 
\end{solution}

\chapter{1983-1984}

\chapter{1984-1985}

\begin{exercise}
Имеется 68 монет, причем известно, что любые две монеты различаются по весу. За 100 взвешиваний на \newline двухчашечных весах без гирь найдите самую тяжелую и самую легкую монеты.
\end{exercise}

\begin{solution}
Количество монет намекает, что в первую очередь их надо взвешивать по парам. Взвесим 34 пары монет и получим две группы, которые назовем "Легкая" и "Тяжелая". Очевидно, что самая легкая монета находится в "Легкой" группе. Ее можно найти за 33 взвешивания. Аналогично за 33 взвешивания можно найти самую тяжелую монету в "Тяжелой" группе. Всего $34 + 33 + 33 = 100$ взвешиваний.
\end{solution}

\begin{exercise}
В таблицу $10 \times 10$ нужно записать в каком-то порядке цифры $0, 1, 2, 3, \dots, 9$ так, что каждая цифра встречалась бы 10 раз.
\begin{itemize}
\item Можно ли это сделать так, чтобы в каждой строке и в каждом столбце встречалось не более четырех различных цифр?
\item Докажите, что найдется строка или столбец, в которой (в котором) встречается не меньше четырех различных чисел.
\end{itemize}
\end{exercise}

\begin{solution}
Будем доказывать от противного. Предположим, что в строке или столбце не может быть больше 3 различных чисел. Обозначим количество различных чисел в строке $i$ как $r_{i}$, а количество различных чисел в столбце $i$ - $c_{i}$. Оценим сумму
$$
\sum_{i = 1}^{10} r_{i} + \sum_{i = 1}^{10} c_{i} 
$$
Очевидно она не превышает 60. Теперь посчитаем эту сумму вторым способом. Если число встречается в строке, то оно добавляет 1 к $r_{i}$, а если в столбце то 1 к $c_{i}$. Обозначим суммарное количество строк и столбцов в которых встречается число $i$ как $q_{i}$. Несложно показать, что $q_{i}$ минимален, когда количества занимаемых числом строк и столбцов максимально близки. В данном случае минимум равен 7. Найдем оценку снизу для
$$
\sum_{i = 1}^{10} q_{i}
$$
Очевидно она равна 70. Но выше мы показали, что она не превышает 60. Противоречие.
\end{solution}

\chapter{1985-1986}

\chapter{1986-1987}

\begin{exercise}
Берутся всевозможные непустые подмножества из множества чисел $1, 2, 3, \dots, N$. Для каждого подмножества берется величина, обратная к произведению всех его чисел. Найдите сумму всех таких обратных величин.
\end{exercise}

\begin{solution}
Несложно заметить, что требуемая величина представляется в виде
$$
\left(1 + \frac{1}{1}\right)\left(1 + \frac{1}{2}\right)\left(1 + \frac{1}{3}\right)\dots \left(1 + \frac{1}{N}\right) - 1 = N
$$
\end{solution}

\chapter{1995-1996}

\begin{exercise}
Докажите, что существует бесконечно много таких троек чисел $n-1$, $n$, $n+1$, что:
\begin{enumerate}
\item $n$ представимо в виде суммы двух квадратов натуральныхчисел, a $n - 1$ и $n + 1$ - нет;
\item каждое из трех чисел представимо в виде суммы двух квадратов натуральных чисел.
\end{enumerate}
\end{exercise}

\chapter{1997-1998}

\begin{exercise}
Шахматный король обошел всю доску $8 \times 8$, побывав на каждой клетке по одному разу, вернувшись последним ходом в исходную клетку. Докажите, что он сделал четное число диагональных ходов.
\end{exercise}

\begin{solution}
Используем шахматную раскраску. При ходе по диагонали цвет поля на котором стоит король не меняется. Так как король вернулся обратно, то для каждого хода в одну сторону должен найтись противоположный ход. То есть для каждой хода с черной клетки на белую должен найтись ход с белой клетки на черную. Значит ходов меняющих цвет клетки четное число. Всего король сделал 64 хода, а значит диагональных ходов тоже четное число. 
\end{solution}

\chapter{1990-1991}

\begin{exercise}
Квадрат $8 \times 8$ клеток выкрашен в белый цвет. Разрешается выбрать в нем любой прямоугольник из трех клеток и перекрасить все их в противоположный цвет (белые в черный, черные - в белый). Удастся ли несколькими такими операциями перекрасить весь квадрат в черный цвет?
\end{exercise}

\begin{solution}
Раскрасим квадрат периодически в три цвета.
$$
\begin{matrix}
1 & 2 & 3 & 1 & 2 & 3 & 1 & 2 \\
2 & 3 & 1 & 2 & 3 & 1 & 2 & 3 \\
3 & 1 & 2 & 3 & 1 & 2 & 3 & 1 \\
1 & 2 & 3 & 1 & 2 & 3 & 1 & 2 \\
2 & 3 & 1 & 2 & 3 & 1 & 2 & 3 \\
3 & 1 & 2 & 3 & 1 & 2 & 3 & 1 \\ 
1 & 2 & 3 & 1 & 2 & 3 & 1 & 2 \\
2 & 3 & 1 & 2 & 3 & 1 & 2 & 3 
\end{matrix}
$$
Заметим, что за каждую операцию перекрашивается одна цифра $1$ и одна цифра $2$. Но цифр $2$ на одну больше чем цифр $1$. 
\end{solution}

\chapter{2000-2001}

\begin{exercise}
В весеннем туре турнира городов 2000 года старшеклассникам страны $N$ было предложено 6 задач. Каждую задачу решили ровно 1000 школьников, но никакие два школьника не решили вместе все 6 задач. Каково наименьшее возможное число старшеклассников страны $N$, принявших участие в весеннем туре?
\end{exercise}

\begin{exercise}
Натуральное число $n$ разрешается заменить на число $ab$, если $a + b = n$ и числа $a$ и $b$ натуральные. Можно ли с помощью таких замен получить из числа 22 число 2001?
\end{exercise}

\begin{solution}
$$
22 \xrightarrow{22 = 21 + 1} 21 \xrightarrow{21 = 20 + 1} 20 \xrightarrow{20 = 19 + 1} 19 \xrightarrow{19 = 18+1} 18 \xrightarrow {18 = 11 + 7} 77 \xrightarrow{77 = 67 + 10} 670 \xrightarrow{670=667+3} 2001
$$
\end{solution}

\chapter{2008-2009}

\begin{exercise}
В ячейку памяти компьютера записали число 6. Далее компьютер делает миллион шагов. На шаге номер $n$ он увеличивает число в ячейке на наибольший общий делитель этого числа и $n$. Докажите, что на любом шаге компьютер увеличивает число в ячейке либо на 1, либо на простое число.
\end{exercise}

\begin{solution}

\end{solution}

\chapter{2012-2013}

\begin{exercise}
Числа $1, 2, \dots, 100$ стоят по кругу в некотором порядке. Может ли случиться, что у любых двух соседних чисел модуль разности не меньше 30, но не больше 50?
\end{exercise}

\begin{solution}
Разделим числа на две равные по размеру группы. В первую группу поместим числа от $n$ до $n + 49$, где $n$ выберем позже. Во вторую группу поместим все остальные числа. Выберем $n$ так, чтобы разность между любыми двумя числами во второй группе была либо меньше 30, либо больше 50. Условия для этого - $100 - (n + 50) < 30$ и $n - 2 < 30$. Значит подходит $20 < n < 32$. Выберем, например, $n = 22$. Итак, в первой группе числа от 22 до 71, а во второй остальные. Числа второй группы не могут соседствовать друг с другом, а значит они должны чередоваться с числами первой группы. Но с числом 22 может стоять только одно число второй группы - это число 72. Получили противоречие.
\end{solution}

\chapter{2015-2016}

\begin{exercise}
Из целых чисел от 1 до 100 удалили $k$ чисел. Обязательно ли среди оставшихся чисел можно выбрать $k$ различных чисел с суммой 100, если
\begin{itemize}
\item $k = 9$
\item $k = 8$
\end{itemize}
\end{exercise}

\begin{solution}
Пусть удалены числа от 1 до 9 включительно. Тогда минимальная сумма любых 9 из оставшихся чисел будет не меньше 
$$
\sum_{i = 0}^{8} (10 + i) = 126
$$
Поэтому для $k = 9$ ответ нет. Во втором случае рассмотрим пары чисел дающих в сумме 25. 1 и 24, 2 и 23, $\dots$, 12 и 13. Таких пар всего 12. Значит в любом случае после удаления останется не менее 4 пар. Сумма чисел в которых будет 100.
\end{solution}

\begin{exercise}
Дан клетчатый квадрат $10 \times 10$. Внутри него провели 80 единичных отрезков по линиям сетки, которые разбили квадрат на 20 многоугольников равной площади. Докажите, что все эти многоугольники равны.
\end{exercise}

\begin{solution}
Площадь каждого многоугальника равна $\frac{100}{20} = 5$. С другой стороны сумма периметров всех многоугольников равна
$$
\sum_{i=1}^{20} P_{i} = 80 \cdot 2 + 10 \cdot 4 = 200
$$
Значит средний периметр равен $\frac{200}{20} = 10$. Но и минимальный периметр многоугольника из 5 клеток тоже равен 10. И такой многоугольник только один - квадрат с дополнительной клеткой. Значит все многоугольники имеют одну и ту же форму.
\end{solution}

\chapter{2019-2020}

\begin{exercise}
На доске написаны $2n$ последовательных целых чисел. За ход можно разбить написанные числа на пары \newline произвольным образом и каждую пару чисел заменить на их сумму и разность (не обязательно вычитать из большего числа меньшее, все замены происходят одновременно). Докажите, что на доске больше никогда не появятся $2n$ последовательных чисел.
\end{exercise}

\chapter{2020-2021}

\begin{exercise}
Существуют ли 100 таких натуральных чисел, среди которых нет одинаковых, что куб одного из них равен сумме кубов остальных?
\end{exercise}

\chapter{2022-2023}

\begin{exercise}
На клетчатой доске $10 \times 10$ в одной из клеток сидит бактерия. За один ход бактерия сдвигается в соседнюю по стороне клетку и делится на две бактерии (обе остаются в той же клетке). Затем снова одна из сидящих на доске бактерий сдвигается в соседнюю по стороне клетку и делится на две, и так далее. Может ли после нескольких таких ходов во всех клетках оказаться поровну бактерий?
\end{exercise}

\begin{exercise}
Назовём натуральное число заурядным, если в его десятичной записи встречаются
только нули и единицы. Пусть произведение двух заурядных чисел оказалось заурядным числом. Обязательно ли тогда сумма цифр произведения равна произведению сумм цифр сомножителей?
\end{exercise}

\begin{exercise}
Натуральные числа от 1 до 100 раскрашены в три цвета: 50 чисел - в красный, 25
чисел - в жёлтый и 25 чисел - в зелёный. Известно, что все красные и жёлтые числа можно разбить на 25 троек так, чтобы в каждой тройке было два красных числа и одно жёлтое, которое больше одного красного и меньше другого. Аналогичное утверждение верно для красных и зелёных чисел. Обязательно ли все 100 чисел можно разбить на 25 четвёрок, в каждой из которых два красных числа, одно жёлтое и одно зелёное, при этом жёлтое и зелёное числа лежат между красными?
\end{exercise}

\begin{solution}
Обозначим $i$-тое по порядку желтое число - $y_{i}$, зеленое - $g_{i}$, а красные разделим на две группы. Первые по порядку 25 - $r_{i}$, а оставшиеся - $R_{i}$. Очевидно $y_{i} > r_{i}$, в ином случае для какого-то желтого не нашлось бы меньшего красного. Аналогично - $r_{i} < g_{i}$. Те же рассуждения доказывают, что $y_{i} < R_{i}$ и $g_{i} < R_{i}$. Значит мы можем выбрать в качестве $i$-той четверки $r_{i}$, $g_{i}$, $y_{i}$, $R_{i}$.
\end{solution}

\chapter{2023-2024}

\begin{exercise}
На урок физкультуры пришло 12 детей, все разной силы. Физрук 10 раз делил их на две команды
по 6 человек, каждый раз новым способом, и проводил состязание по перетягиванию каната.
Могло ли оказаться, что все 10 раз состязание закончилось вничью (то есть суммы сил детей в
командах были равны)?
\end{exercise}

\begin{solution}
Построим пример. Допустим силы детей будут равны $1, 2, 3, 4, \dots, 12$. Рассмотрим 6 пар $(1, 12)$, $(2, 11)$, $(3, 10)$, $(4, 9)$, $(5, 8)$, $(6, 7)$. Сумма в каждой паре равна 13. В одну команду мы можем взять любые три пары, а в другую оставшиеся три пары. И команды будут равны по силе. Количество таких способов $\frac{1}{2} \binom{6}{3} = 10$.
\end{solution}

\begin{exercise}
Докажите, что среди вершин любого выпуклого девятиугольника можно найти три, образующие
тупоугольный треугольник, ни одна сторона которого не совпадает со сторонами девятиугольника.
\end{exercise}

\begin{exercise}
Имеется кучка из 100 камней. Играют двое. Первый берёт 1 камень, потом второй берёт 1 или
2 камня, потом первый берёт 1, 2 или 3 камня, затем второй 1, 2, 3 или 4 камня и так далее.
Выигрывает взявший последний камень. Кто может обеспечить себе победу, как бы ни играл его
соперник?
\end{exercise}

\begin{solution}
Выигрывает первый. Рассмотрим его стратегию. Представим 100 как сумму $1 + 3 + 5 + 7 + \dots + 19$. Пусть первый всегда дополняет количество взятых вторым на ходе с номером $k$ камней до $k$-того слагаемого этой суммы. То есть первый и второй на $k$-том ходе берут в сумме $2k + 1$ камней и ход заканчивает первый. Тогда, как нетрудно заметить, последние оставшиеся камни берет первый игрок.
\end{solution}

\begin{exercise}
В каждой клетке таблицы $N \times N$ записано число. Назовём клетку $C$ хорошей, если в какой-то из клеток, соседних с $C$ по стороне, стоит число на 1 больше, чем в $C$, а в какой-то другой из клеток, соседних с $C$ по стороне, стоит число на 3 больше, чем в $C$. Каково наибольшее возможное количество хороших клеток?
\end{exercise}

\begin{exercise}
На каждой из 99 карточек написано действительное число. Все 99 чисел различны, а их общая сумма иррациональна. Стопка из 99 карточек называется неудачной, если для каждого $k$ от 1 до 99 сумма чисел на $k$ верхних карточках иррациональна. Петя вычислил, сколькими способами можно сложить исходные карточки в неудачную стопку. Какое наименьшее значение он мог получить?
\end{exercise}

\end{document} 
