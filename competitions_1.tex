\documentclass[11pt, a4paper]{template}

\begin{document}

\begin{titlepage}
  \begin{center}
	\Huge{Задачи журнала Квант}
  \end{center}
\end{titlepage}

\chapter{1970}

\begin{exercise}[M24]
Любую дробь $\frac{m}{n}$, где $m$ и $n$ - натуральные числа, $1 < m < n$, можно представить в виде суммы нескольких дробей вида $\frac{1}{q}$, причём таких, что знаменатель каждой следующей из этих дробей делится на знаменатель предыдущей дроби. Докажите это.
\end{exercise}

\begin{solution}
По условию задачи
$$
\frac{m}{n} = \frac{1}{a_{1}} + \frac{1}{a_{1}a_{2}} + \frac{1}{a_{1}a_{2}a_{3}} + \dots + \frac{1}{a_{1}a_{2}\dots a_{k}} = \frac{1}{a_{1}} \left( 1 + \frac{1}{a_{2}} \left( 1 + \frac{1}{a_{3}} (1 + \dots ) \right) \right)
$$
Исходя из этого можно предложить следующий алгоритм. Пусть $a_{1}$ будет  минимальным числом при котором выполняется $ma_{1} > n$. Очевидно, что $ma_{1} - n < m$. Тогда 
$$
\frac{ma_{1}}{n} - 1 = \frac{ma_{1}-n}{n} = \frac{m_{1}}{n} < \frac{m}{n}
$$
Аналогично поступая с $\frac{m_{1}}{n}$ находим $a_{2}$. Так как числитель всегда уменьшается и взаимно прост со знаментателем, то в итоге он станет равным единице. На этом алгоритм завершается. 
\end{solution}

\begin{exercise}[M27]
Если
$$
\frac{a}{b-c} + \frac{b}{c-a} + \frac{c}{a-b} = 0
$$
то
$$
\frac{a}{(b-c)^{2}} + \frac{b}{(c-a)^{2}} + \frac{c}{(a-b)^{2}} = 0
$$
\end{exercise}

\begin{exercise}[M33]
Рассмотрим натуральное число $n > 1000$. Найдём остатки от деления числа $2^{n}$ на числа $1, 2, 3, \dots, n$ и найдём сумму всех этих остатков. Докажите, что эта сумма больше $2n$.
\end{exercise}

\begin{solution}
Обозначим остаток от деления $2^{n}$ на $m$ как $d(m)$. Заметим, что $d(2m + 1) \geqslant 1$. А из этого следует, что $d(2^{i}q) \geqslant 2^{i}$, где $q = 2m+1$. Сумма остатков тогда равна
$$
S_{n} = \sum_{m = 1}^{n} d_{m} \geqslant \sum_{i=1}^{p} 2^{i} N_{i} 
$$
Где $N_{i}$ - количество чисел вида $(2m + 1)2^{i}$ не больших $n$.
\end{solution}

\chapter{1971}

\begin{exercise}[M63]
Можно ли из 18 плиток размером $1 \times 2$ выложить квадрат так, чтобы при этом не было ни одного прямого "шва", соeдиняющего противоположные стороны квадрата и идущего по краям плиток?
\end{exercise}

\begin{exercise}[M113]
Для любого натурального числа $n$ существует составленное из цифр 1 и 2 число, делящееся на $2^{n}$. Докажите это. (Например, 2 делится на 2, число 12 делится на 4, на 8 делится число 112, а на 16 делится число 2112.)
\end{exercise}

\begin{solution}
Докажем это по индукции. Допустим, что существует число $S = \overline{a_{1}a_{2}\dots a_{n}}$ из $n$ цифр, которое делится на $2^{n}$. Примеры таких чисел приведены в условии задачи. Тогда образуем следующее число из $n + 1$ цифры $P = 10^{n} d + S$, где $d$ равно один или два. Если $S$ не делится на $2^{n+1}$, то выберем $d = 1$. Тогда $P = 2^{n}(5^{n} + R)$, где $R$ нечетное. И значит $P$ делится на $2^{n+1}$. В ином случае выберем $d = 2$. И получим аналогичный результат.
\end{solution}

\begin{exercise}
Возрастающая последовательность натуральных чисел $a_{1} < a_{2} < a_{3} < \dots$ такова, что каждое натуральное число либо принадлежит ей, либо представимо в виде суммы двух членов последовательности, быть может, одинаковых. Докажите для любого натурального $n$ неравенство $a_{n} \leqslant n^{2}$.
\end{exercise}

\begin{solution}
Рассмотрим возможные суммы двух, быть может одинаковых, членов последовательности $a_{1}, a_{2}, \dots, a_{n-1}$. Их, как несложно посчитать,
$$
\binom{n-1}{2} + n-1 = \binom{n}{2}
$$
Вместе с самими членами этой последовательности
$$
\binom{n}{2} + n-1 < n^{2}
$$ 
Значит какие-то числа от 1 до $n^{2}$ не были представлены. Отсюда следует утверждение задачи.
\end{solution}

\chapter{1972}

\begin{exercise}[M146]
\begin{itemize}
\item В вершинах правильного 7-угольника расставлены чёрные и белые фишки. Докажите, что есть 3 фишки одного цвета, лежащие в вершинах равнобедренного треугольника.
\item Верно ли аналогичное утверждение для 8-угольника?
\item Выясните, для каких правильных $n$-угольников аналогичное верно, а для каких - нет.
\end{itemize}
\end{exercise}

\begin{solution}
В семиугольнике, очевидно, найдется две фишки одного цвета стоящие рядом. Пусть этот цвет - белый. Обозначим их $L$ и $R$ если следовать по часовой стрелке. Тогда фишки $L_{1}$ слева от $L$ и $R_{1}$ справа от $R$ должны быть черного цвета. Но тогда при любом цвете фишки $Q$ напротив стороны $LR$  возникнет равнобедренный треугольник $LRQ$ или $L_{1}R_{1}Q$. Очевидно данное рассуждение справедливо для всех многоугольников с нечетным числом сторон, кроме треугольника. \\
Для восьмиугольника контрпримером является раскраска $11221122$. Очевидно в ней нет одноцветных равнобедренных тругольников. \\
Для произвольного правильного $n$-угольника с четным $n > 8$ любая раскраска где нет рядом стоящих фишек одного цвета, очевидно, не подходит. Значит будет две одноцветных фишки стоящие рядом. Рассуждая по аналогии, получим, что раскраска имеет вид $11221122\dots$. Но в ней в силу периодичности будут равнобедренные тругольники одного цвета.
\end{solution}

\begin{exercise}[M154]
На прямой дано 50 отрезков. Докажите, что верно хотя бы одно из следующих
утверждений:
\begin{itemize}
\item некоторые 8 из этих отрезков имеют общую точку;
\item некоторые 8 из этих отрезков таковы, что никакие два из них не пересекаются.
\end{itemize}
\end{exercise}

\begin{solution}
Рассмотрим отрезок $[L_{1}, R_{1}]$ правый конец которого лежит левее всех остальных правых концов. Если существует 7 отрезков, левые концы которых лежат левее $R_{1}$, то задача решена. В ином случае существует не менее 43 отрезков, левые концы которых лежат правее $R_{1}$. Тем же способом выберем отрезок $[L_{2}, R_{2}]$. Аналогично, если существует 7 отрезков, левые концы которых лежат левее $R_{2}$, то задача решена. Если нет, то существет 36 отрезков левые концы которых лежат правее $R_{2}$. Действуя таким образом получим либо систему из 8 отрезков, имеющих общую точку, либо систему из отрезков $[L_{1}, R_{1}], [L_{2}, R_{2}], \dots, [L_{8}, R_{8}]$ никакие два из которых не пересекаются. 
\end{solution}

\chapter{1973}



\chapter{1974}

\begin{exercise}[M287]
Существует ли такая последовательность натуральных чисел, что любое натуральное число представимо в виде разности двух чисел этой последовательности единственным образом?
\end{exercise}

\begin{solution}
Рассмотрим алгоритм для построения такой последовательности. Пусть последовательность уже построена до члена с номером $k$ и выглядит как $a_{1}, a_{2}, \dots, a_{k}$. Выберем наименьшее натуральное число $S$, которое еще не встречалось как разность между членами искомой последовательности, то есть $S \neq a_{i} - a_{j}$, где $j < i$ и $i, j \leqslant k$. Тогда добавим два новых элемента $a_{k+1} = 2a_{k}$ и $a_{k+2} = 2a_{k} + S$. Заметим, что $a_{k+1} - a_{i} = 2a_{k} - a_{i} \geqslant a_{k}$, где $i \leqslant k$, а значит точно не встречается среди разностей уже построенной последовательности. 
\end{solution}

\chapter{1975}

\begin{exercise}[M306]
Из шахматной доски удалена одна угловая клетка. На какое наименьшее число равновеликих (одинаковых по площади) треугольников можно разрезать оставшуюся часть доски?
\end{exercise}

\begin{exercise}[M335]
\begin{itemize}
\item В квадрате размером $7 \times 7$ клеток отмечены центры $k$ клеток. При этом никакие четыре отмеченные точки не являются вершинами прямоугольника со сторонами, параллельными сторонам квадрата. При каком наибольшем $k$ это возможно?
\item Решите ту же задачу для квадрата размером $13 \times 13$ клеток.
\end{itemize}
\end{exercise}

\chapter{1976}

\begin{exercise}[M387]
Существует ли такое натуральное число, что если приписать его само к себе, то получится точный квадрат? 
\end{exercise}

\begin{solution}
Обозначим искомое число $n$. Тогда задачу можно записать в виде
$$
n(10^{p}+1) = x^{2}
$$
И $n < 10^{p}$. Легко заметить, что если $10^{p} + 1$ содержит любой простой множитель более чем в первой степени, то искомое число нашлось. Докажем, что $10^{11}+1$ подходит. Действительно
$$
10^{11}+1 = 11 \cdot (10^{10} - 10^{9} + 10^{8} - 10^{7} + \dots + 1)
$$
Легко видеть, что выражение в скобках делится на 11. Значит искомое число
$$
n = \frac{10^{11} + 1}{121}
$$ 
\end{solution}

\begin{exercise}[M390]
Существует бесконечно много таких натуральных чисел $n$, что сумма цифр десятичной записи числа $2^{n}$ больше суммы цифр десятичной записи числа $2^{n+1}$. Докажите это.
\end{exercise}

\begin{exercise}[M393]
Найдите сумму
$$
f(0) + f\bigp{\frac{1}{n}} + f\bigp{\frac{2}{n}} + \dots + f\bigp{\frac{n-1}{n}} + f(1) 
$$
где $f(x) = \frac{4^{x}}{4^{x}+2}$.
\end{exercise}

\begin{solution}
Сложим этот ряд с самим собой, но записанным в обратном порядке. Заметим что
$$
f\bigp{\frac{i}{n}} + f\bigp{\frac{n-i}{n}} = \frac{4^{\frac{i}{n}}}{4^{\frac{i}{n}}+2} + \frac{4^{\frac{n-i}{n}}}{4^{\frac{n-i}{n}}+2} = \frac{4 + 2 \cdot 4^{\frac{i}{n}} + 4 + 2 \cdot 4^{\frac{n-i}{n}}}{4 + 2 \cdot 4^{\frac{i}{n}} + 2 \cdot 4^{\frac{n-i}{n}} + 4} = 1
$$
То есть, удвоенная сумма ряда равна $n + 1$. Значит ответ $\frac{n+1}{2}$.
\end{solution}

\chapter{1977}

\begin{exercise}[M425]
Существует ли такое натуральное $n$, что каждое рациональное число между нулём и единицей представимо в виде суммы $n$ чисел, обратных натуральным?
\end{exercise}

\begin{exercise}[M466]
Среди 1977 монет 50 фальшивых. Каждая фальшивая монета отличается от настоящей на один грамм (в ту или в другую сторону). Имеются чашечные весы со стрелкой, показывающей разность масс грузов на чашках. За одно взвешивание про одну выбранную монету нужно узнать, фальшивая она или настоящая. Научитесь это делать!
\end{exercise}

\chapter{1978}

\chapter{1979}

\begin{exercise}[M554]
Назовём натуральное число $n$ хорошим, если существуют (не обязательно различные) такие натуральные числа $a_{1}, a_{2}, \dots, a_{k}$, что $a_{1} + a_{2} + \dots + a_{k} = n$ и $\frac{1}{a_{1}} + \frac{1}{a_{2}} + \dots + \frac{1}{a_{k}} = 1$. Известно, что все числа между 33 и 73 - хорошие. Докажите, что все числа, большие 73, - тоже хорошие.
\end{exercise}

\begin{solution}
Докажем несколько простых лемм. Во первых, если $n$ - хорошее, то $2n + 2$ тоже хорошее. 
$$
\sum_{i = 1}^{k} a_{i} = n \implies 2 + \sum_{i = 1}^{k} 2a_{i} = 2n + 2, \quad \sum_{i = 1}^{k} \frac{1}{a_{i}} = 1 \implies \frac{1}{2} + \sum_{i = 1}^{k} \frac{1}{2a_{i}} = 1
$$
Аналогично докажем, что $2n+9$ тоже хорошее
$$
\sum_{i = 1}^{k} a_{i} = n \implies 3 + 6 + \sum_{i = 1}^{k} 2a_{i} = 2n + 9, \quad \sum_{i = 1}^{k} \frac{1}{a_{i}} = 1 \implies \frac{1}{3} + \frac{1}{6} + \sum_{i = 1}^{k} \frac{1}{2a_{i}} = 1
$$
Далее очевидно по индукции.
\end{solution}

\begin{exercise}[M557]
Среди любых $n$ попарно взаимно простых чисел, больших 1 и меньших $(2n - 1)^{2}$, есть хотя бы одно простое число. Докажите это.
\end{exercise}

\begin{solution}
Выберем произвольное множество $A$, удовлетворяющее условиям задачи. Допустим простых чисел в $A$ нет. Значит каждое число $a_{i}$ в $A$ делится, как минимум, на два простых числа $p_{i}$ и $q_{i}$, где $i = 1, \dots, n$. Пусть $p_{i} \geqslant q_{i}$.  Так как любые два числа в $A$ взаимно просты, то $q_{1}, q_{2}, \dots, q_{n}$ все различны. Тогда $q_{i}^{2} \leqslant p_{i} q_{i} = a_{i} < (2n - 1)^{2}$. Значит $q_{i} < 2n - 1$. Но проcтых чисел меньших $2n - 1$ и больших $1$ не более $n - 1$. Противоречие.
\end{solution}

\chapter{1991}

\begin{exercise}[M1275]
Последовательность $a_{1}, a_{2}, a_{3}, \dots$, натуральных чисел такова, что при любом натуральном $n$ верно равенство $a_{k+2} = a_{k} a_{k+1} + 1$. Докажите, что при $n > 9$ число $a_{n} - 22$ составное.
\end{exercise}

\begin{solution}
Заметим, что если $a_{k} = 0$, то последовательность выглядит следующим образом
$$
a_{k-1}, 0, 1, 1, 2, 3, 7, 22, \dots 
$$
Но мы можем получить эту последовательность представляя, что это значения по модулю $a_{k}$. Значит 
$$
a_{k+6} = 22 \mod a_{k} \implies a_{k+6} - 22 = 0 \mod a_{k}
$$
\end{solution}

\begin{exercise}

\end{exercise}

\chapter{1999}

\begin{exercise}[M1667]
Натуральный ряд разбит на два бесконечных множества чисел. Докажите, что сумма некоторых 100 чисел одного из этих множеств равна сумме некоторых 100 чисел другого множества.
\end{exercise}

\chapter{2000}

\begin{exercise}[M1731]
В ряд нарисованы 60 звёздочек. Два игрока по очереди заменяют звёздочки на цифры. Какую именно из оставшихся звёздочек заменять на цифру, решает игрок, делающий очередной ход. Докажите, что второй игрок может играть так, чтобы полученное число (возможно, начинающееся на цифру 0) делилось на 13.
\end{exercise}

\begin{solution}
Заметим, что любое число вида $\overline{abcabc} = 1001 \cdot \overline{abc}$, а значит делится на 13. Значит можно разбить зведочки на 10 блоков по 6 звездочек, а второй игрок может дублировать цифру написанную первым игроком, чтобы получился набор блоков указанного выше типа. 
\end{solution}

\chapter{2001}

\begin{exercise}[M1756]
Среди любых трёх из нескольких данных натуральных чисел можно выбрать два, одно из которых делится на другое. Докажите, что все числа можно так покрасить двумя красками, что из любых чисел одного цвета одно делится на другое.
\end{exercise}

\chapter{2002}

\begin{exercise}
Назовём несоседние натуральные числа $a$ и $b$ близкими, если $a^{2} - 1$ делится на $b$ и $b^{2} - 1$ делится на $a$.
\begin{enumerate}
\item Пусть $n > 1$. Докажите, что на отрезке $[n; 8n - 8]$ существует пара близких чисел.
\item Укажите такое $n > 1$, что на отрезке $[n; 8n - 9]$ нет ни одной пары близких чисел.
\end{enumerate} 
\end{exercise}

\chapter{2003}

\begin{exercise}[M1847]
В 8 банках сидят 80 пауков. Разрешено выбрать любые две банки, суммарное число пауков в которых чётное, и пересадить часть пауков из одной банки в другую, чтобы их стало поровну. При любом ли начальном распределении пауков в банках с помощью нескольких таких операций можно добиться того, чтобы в банках оказалось поровну пауков?
\end{exercise}

\chapter{2006}

\begin{exercise}[M1981]
В клетках таблицы размером $11 \times 11$ расставлены натуральные числа от 1 до 121. Дима посчитал произведение чисел в каждой строке, а Саша — произведение чисел в каждом столбце. Могли ли они получить одинаковые наборы из 11 чисел?
\end{exercise}

\begin{solution}
Рассмотрим простые числа из промежутка от 1 до 121, для которых в таблице нет кратных. Это 61, 67, 71, 73, 79, 83, 89, 97, 101, 103, 107, 109, 113. Их 13 штук. Значит в каком то столбце их не менее двух. Но если эти два числа встретились в одном столбце, то одно из них или его кратное должно быть и в одной строке с другим. Но этого быть не может.  
\end{solution}

\chapter{2007}

\begin{exercise}[M2033]
У ведущего имеется колода из 52 карт. Зрители хотят узнать, в каком порядке лежат карты (не уточняя, сверху вниз или снизу вверх). Разрешено задавать ведущему вопросы вида "Сколько карт лежит между такой-то и такой-то картами?". Один из зрителей знает, в каком порядке лежат карты. Какое наименьшее число вопросов он должен задать, чтобы остальные зрители по ответам на эти вопросы могли узнать порядок карт в колоде?
\end{exercise}

\chapter{2008}

\begin{exercise}[M2096]
Депутаты парламента образовали 2008 комиссий, каждая - не более чем из 10 человек. Любые 11 комиссий имеют хотя бы одного общего члена. Докажите, что существует человек, входящий во все комиссии.
\end{exercise}



\chapter{2010}

\begin{exercise}[M2162]
Семизначный код, состоящий из семи различных цифр, назовём хорошим. Паролем сейфа является хороший код. Сейф откроется, если введён хороший код и на каком-нибудь месте цифра кода совпала с соответствующей цифрой пароля. За какое наименьшее количество попыток можно с гарантией открыть сейф?
\end{exercise}

\begin{exercise}[2202]
Дана арифметическая прогрессия из 22 различных натуральных чисел, каждое из которых является точной степенью (то есть степенью натурального числа, большей 1). Докажите, что разность этой прогрессии больше 2010.
\end{exercise}

\chapter{2020}

\begin{exercise}[M2593]
Каждая вершина правильного многоугольника окрашена в один из трех цветов так, что в каждый из трех цветов окрашено нечетное число вершин. Докажите, что количество равнобедренных треугольников, вершины которых окрашены в три разных цвета, нечетно.
\end{exercise}

\begin{exercise}[M2601]
Глеб задумал натуральные числа $N$ и $a$, где $a < N$. Число $a$ он написал на доске. Затем Глеб стал проделывать такую операцию: делить $N$ с остатком на последнее выписанное на доску число и полученный остаток от деления также записывать на доску. Когда на доске
появилось число 0, он остановился. Мог ли Глеб изначально выбрать такие $N$ и $a$,
чтобы сумма выписанных на доске чисел была больше $100N$?
\end{exercise}

\begin{exercise}[2627]
Дана бесконечная арифметическая прогрессия. Рассматриваются произведения пар ее членов. Докажите, что произведения в каких-то двух различных парах отличаются не более чем на 1.
\end{exercise}

\begin{solution}
Зададим прогрессию - $a_{n} = a_{0} + nd$. Легко доказать, что найдется бесконечно много $a_{m}$ делящихся нацело на $a_{0}$. Для этого запишем
$$
a_{m} = a_{0} + md = pa_{0} \implies (p-1)a_{0} = md
$$
Тогда положив $p = s \cdot d+1$ и $m = s \cdot a_{0}$ получим верное равенство. Теперь запишем отношение двух членов прогрессии
$$
\frac{a_{0} + nd}{a_{0} + md} = \frac{\frac{n}{m}(a_{0}+md) + a_{0} - \frac{n}{m}a_{0}}{a_{0} + md} = \frac{n}{m} + \frac{a_{0} - \frac{n}{m}a_{0}}{a_{0} + md}
$$
Пусть $n$ нацело делится на $m$ и $n = q \cdot m$. Тогда мы переписываем выражение выше в виде
$$
\frac{n}{m} + \frac{a_{0} - \frac{n}{m}a_{0}}{a_{0} + md} = q + \frac{a_{0}(1 - q)}{a_{0} + md}
$$
Выбираем $m$ так, чтобы $a_{0} + md = pa_{0}$, а $q = p+1$. Тогда
$$
q + \frac{a_{0}(1 - q)}{a_{0} + md} = q + \frac{a_{0}(1 - p - 1)}{pa_{0}} = q - 1
$$
Выбрав другое $a_{0}$ мы повторим процедуру и получим равенство
$$
\frac{a_{n}}{a_{m}} = \frac{a_{N}}{a_{M}} \implies a_{n}a_{M} - a_{m}a_{N} = 0
$$
\end{solution}

\chapter{2021}

\begin{exercise}[M2638]
Существует ли целое положительное число $n$ такое, что все его цифры (в десятичной записи) больше 5, а все цифры числа $n^{2}$ меньше 5?
\end{exercise}

\begin{exercise}[M2669]
Докажите, что для любого натурального числа $n$ числа $1, 2, \dots, n$ можно разбить на несколько групп так, чтобы сумма чисел в каждой группе была равна степени тройки.
\end{exercise}

\begin{exercise}[M2673]
В очереди на посадку в $n$-местный самолет стоят $n$ пассажиров. Первой в очереди стоит рассеянная старушка, которая, зайдя в самолет, садится на случайно выбранное место. Каждый следующий пассажир садится на свое место, если оно свободно, и на случайное место в противном случае. Сколько в среднем пассажиров окажутся не на своих местах?
\end{exercise}

\end{document} 
