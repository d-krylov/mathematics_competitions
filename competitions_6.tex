\documentclass[11pt, a4paper]{template}

\begin{document}

\begin{titlepage}
  \begin{center}
	\Huge{Санкт-Петербургская олимпиада по математике}
  \end{center}
\end{titlepage}

\chapter{1961}

\begin{exercise}
Докажите, что все числа вида $1156, 111556, 11115556, \dots$ являются точными квадратами.
\end{exercise}

\begin{solution}
Легко заметить, что все числа этого вида представляются в форме
$$
\frac{10^{n}-1}{9} \cdot 10^{n} + \frac{10^{n} - 1}{9} \cdot 5 + 1 = \frac{10^{2n} + 4 \cdot 10^{n} + 4}{9} = \frac{(10^{n} + 2)^{2}}{3^{2}}
$$
\end{solution}

\chapter{1962}

\begin{exercise}
Докажите, что шахматную доску $201 \times 201$ можно обойти ходом шахматного коня, побывав на каждом поле ровно один раз.
\end{exercise}

\begin{exercise}
Освободитесь от иррациональности в знаменателе
$$
\frac{1}{\sqrt[3]{a} + \sqrt[3]{b} + \sqrt[3]{c}}
$$
\end{exercise}

\begin{exercise}
Дан многогранник. Число сторон всех его граней, кроме одной, делится на данное натуральное $n$, большее 1. Докажите, что грани этого многогранника нельзя раскрасить двумя красками так, чтобы соседние грани были окрашены в разные цвета.
\end{exercise}

\begin{solution}
Пусть многогранник окрашен в черный и белый цвет. Пусть единственная грань, число сторон которой не делится на $n$, будет окрашена в белый. Посчитаем количество ребер соединяющих черную и белую грани. С одной стороны
$$
E = \sum_{i = 1}^{N_{w}} n\cdot q_{i} + q = q \mod n
$$
С другой стороны
$$
E = \sum_{i = 1}^{N_{b}} n\cdot p_{i} = 0 \mod n
$$
Противоречие.
\end{solution}

\begin{exercise}
Рассматриваются бесконечные последовательности, состоящие из чисел $1, 2, \dots, N$. Два члена в двух последователь-ностях назовем одинаковыми, если они равны и имеют одинаковые номера. Сколько можно составить последователь-ностей так, чтобы любые две из них имели ровно $k$ одинаковых членов?
\end{exercise}

\chapter{1968}

\begin{exercise}
Найдите такое целое $n$, что среди цифр десятичной записи числа $5^{n}$ есть по крайней мере 1968 нулей подряд.
\end{exercise}

\chapter{1969}

\begin{exercise}
На площади собралось 50 гангстеров. Они одновременно стреляют друг в друга, причем каждый стреляет в ближайшего к нему или в одного из ближайших. Каково минимальное возможное количество убитых?
\end{exercise}

\chapter{1970}

\begin{exercise}
В квадрате $5 \times 5$ закрашено 16 клеток. Докажите, что в нем можно выбрать квадрат $2 \times 2$, в котором закрашено не менее трех клеток.
\end{exercise}

\begin{exercise}
Положительные числа $a$, $b$, $c$, $d$ таковы, что $a^{2} + b^{2} = c^{2} + d^{2}$ и $a^{3} + b^{3} = c^{3} + d^{3}$. Докажите, что $ab = cd$.
\end{exercise}

\begin{solution}

\end{solution}

\begin{exercise}
Натуральные числа $a_{1}, a_{2}, \dots$, таковы, что $0 < a_{1} < а_{2} < \dots$ и $a_{n} < 2n$ для любого $n$. Докажите, что любое натуральное число можно представить в виде разности двух чисел из этой последовательности или как число из самой последовательности.
\end{exercise}

\chapter{1971}

\begin{exercise}
Даны числа $5^{1971}$ и $2^{1971}$. Они записаны подряд. Каково количество цифр полученного числа?
\end{exercise}

\begin{exercise}
Докажите, что уравнение $X^{3} + Y^{3} + Z^{3} = 2$ имеет бесконечно много решений в целых числах.
\end{exercise}

\begin{solution}
Рассмотрим тождество
$$
(1+x)^{3} + (1-x)^{3} - 6x^{2} = 2
$$
Тогда положив $x = 6s^{3}$ получим параметризацию
$$
(1+6s^{3})^{3} + (1-6s^{3})^{3} - (6s^{2})^{3} = 2
$$ 
\end{solution}

\chapter{1972}

\begin{exercise}
Простое число $p$ не равно 3. Докажите, что число $4p^{2} + 1$ можно представить в виде суммы трех квадратов натуральных чисел.
\end{exercise}

\begin{solution}
На самом деле условие выполняется для любых чисел вида $3n+1$ или $3n+2$. Действительно для $p = 3n+1$,
$$
4p^{2} + 1 = 4(3n+1)^{2} + 1 = 36n^{2}+24n+5 = (4n+2)^{2}+(4n+1)^{2}+(2n)^{2}
$$
А для $p = 3n + 2$,
$$
4p^{2} + 1 = 4(3n+2)^{2} + 1 = 36n^{2} + 48n + 17 = (4n + 3)^{2} + (4n + 2)^{2} + (2n + 2)^{2}
$$
\end{solution}

\begin{exercise}
Подряд написано 99 девяток. Докажите, что справа к ним можно приписать 100 цифр так, чтобы получившееся число было точным квадратом.
\end{exercise}

\chapter{1973}

\begin{exercise}
Докажите, что $2^{10} + 5^{12}$ - составное число.
\end{exercise}

\begin{exercise}
Играют двое. Один задумывает десятизначное число, а второй может спрашивать у него о том, какие цифры стоят на конкретном наборе мест в записи. Первый отвечает ему, но без указания того, какие именно цифры на каких именно местах стоят. За какое минимальное число вопросов заведомо можно отгадать число?
\end{exercise}

\chapter{1974}

\begin{exercise}
Известно, что
$$
a+b+c = 7, \quad \frac{1}{a+b}+\frac{1}{b+c}+\frac{1}{a+c} = \frac{7}{10}
$$
Найдите
$$
\frac{a}{b+c}+\frac{b}{a+c}+\frac{c}{a+b}
$$
\end{exercise}

\begin{solution}
$$
\frac{a}{b+c}+\frac{b}{a+c}+\frac{c}{a+b} = (a+b+c)\left( \frac{1}{a+b}+\frac{1}{b+c}+\frac{1}{a+c} \right) - 3 = \frac{19}{10}
$$
\end{solution}

\begin{exercise}
Существует ли 20-значное число, являющеееся точным квадратом, десятичная запись которого начинается с 11 единиц?
\end{exercise}

\chapter{1978}

\begin{exercise}
Пятизначное число делится на 41. Докажите, что любое пятизначное число, полученное из него круговой перестанов-кой цифр, также делится на 41.
\end{exercise}

\begin{exercise}
Клетки доски $100 \times 100$ окрашены в четыре цвета, причем клетки, окрашенные в один цвет, не имеют общих вершин. Докажите, что клетки в углах доски окрашены в разные цвета.
\end{exercise}

\begin{solution}

\end{solution}

\chapter{1979}

\begin{exercise}
Натуральные числа $a_{1}, a_{2}, \dots, a_{n}$, где $n \geqslant 12$, все большие 1 и меньшие $9n^{2}$, попарно взаимно просты. Докажите, что среди них найдется простое число.
\end{exercise}

\chapter{1982}

\begin{exercise}
Клетки прямоугольника $5 \times 41$ раскрашены в два цвета. Докажите, что можно выбрать три строки и три столбца так, что все девять клеток, находящиеся на их пересечении, будут иметь один цвет.
\end{exercise}

\chapter{1983}

\begin{exercise}
Две последовательности чисел $(x_{n})$ и $(y_{n})$ построены так, что $x_{1} = x_{2} = 10$, $y_{1} = y_{2} = -10$ и
$$
x_{n+2}=(x_{n}+1)x_{n+1}+1, \quad y_{n+2}=(y_{n+1} + 1)y_{n}+1
$$
при всех натуральных $n$. Докажите, что любое натуральное число встречается не более чем в одной из этих последовательностей.
\end{exercise}

\begin{exercise}
В правильном 20-угольнике отметили девять вершин. Докажите, что найдется равнобедренный треугольник с вершинами в отмеченных точках.
\end{exercise}

\begin{solution}
Разделим вершины 20-угольника на 4 множества так, что каждое множество является правильным пятиугольником. В пятиугольнике нельзя отмечать более 2 вершин, а иначе образуется равносторонний треугольник. Но тогда нельзя отметить более $2 \cdot 4 = 8$ вершин. Противоречие.
\end{solution}

\chapter{1986}

\chapter{1990}

\begin{exercise}
На экране компьютера - число 123. Компьютер каждую минуту прибавляет к числу на экране 102. Программист Федя в любой момент может изменить число на экране, переставив произвольным образом его цифры. Может ли Федя действовать так, чтобы на экране всегда оставалось трехзначное число?
\end{exercise}

\chapter{1991}

\begin{exercise}
Можно ли разбить числа $1, 2, 3, \dots, 100$ на три группы так, чтобы в первой группе сумма чисел делилась на 102, во второй - на 203, а в третьей - на 304?
\end{exercise}

\begin{exercise}
Дано 70 различных натуральных чисел, не превосходящих 200. Докажите, что какие-то два из них различаются на четыре, пять или девять.
\end{exercise}

\begin{exercise}
Докажите, что число $512^{3} + 675^{3} + 720^{3}$ - составное.
\end{exercise}

\begin{exercise}
С натуральным числом разрешается делать следующие две операции:
\begin{itemize}
\item умножать его на любое натуральное число;
\item вычеркивать в его десятичной записи нули.
\end{itemize}
Докажите, что при помощи этих операций из любого числа можно получить однозначное число.
\end{exercise}

\chapter{1992}

\begin{exercise}
Клетки квадрата $7 \times 7$ раскрашены в два цвета. Докажите, что найдется по крайней мере 21 прямоугольник с вершинами в центрах клеток одного цвета и со сторонами, параллельными сторонам квадрата.
\end{exercise}

\begin{exercise}
На доске написано 128 единиц. За ход можно заменить пару чисел $a$ и $b$ на число $ab+1$. Пусть $A$ - максимальное число, которое может получиться на доске после 127 таких операций. Какова его последняя цифра?
\end{exercise}

\chapter{1993}



\end{document} 
