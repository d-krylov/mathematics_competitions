\documentclass[11pt, a4paper]{template}

\begin{document}

\begin{titlepage}
  \begin{center}
	\Huge{Задачи сайта problems.ru}
  \end{center}
\end{titlepage}

\chapter{Инварианты и полуинварианты}

\begin{exercise}
Есть три печатающих автомата. Первый по карточке с числами $a$ и $b$ выдает карточку с числами $a + 1$ и $b + 1$; второй по карточке с четными числами $a$ и $b$ выдает карточку с числами $\frac{a}{2}$ и $\frac{b}{2}$; третий автомат по паре карточек с числами $a$, $b$ и $b$, $c$ выдает карточку с числами $a$, $c$. Все автоматы возвращают заложенные в них карточки. Можно ли с помощью этих автоматов из карточки $(5, 19)$ получить карточку $(1, 1988)$?
\end{exercise}

\begin{solution}
Заметим, что разность чисел в исходной паре $(5, 19)$ делится на 7. А также то, что все из вышеуказанных преобразований сохраняют это свойство. Так как 1987 не делится на 7, то получить требуемую карточку невозможно. 
\end{solution}

\begin{exercise}
Дно прямоугольной коробки вымощено плитками $1 \times 4$ и $2 \times 2$. Плитки высыпали из коробки и одна плитка $2 \times 2$ потерялась. Ее заменили на плитку $1 \times 4$. Докажите, что теперь дно коробки вымостить не удастся.
\end{exercise}

\begin{solution}
Рассмотрим раскраску вида 
$$
\begin{matrix}
1 & 2 & 1 & 2 & \dots \\
3 & 4 & 3 & 4 & \dots \\
1 & 2 & 1 & 2 & \dots \\
3 & 4 & 3 & 4 & \dots \\
\vdots & \vdots & \vdots & \vdots & \ddots
\end{matrix}
$$
Любая плитка $2 \times 2$ содержит 1 клетку цвета 1. Значит четность плиток этого вида должна быть равна четности клеток цвета 1. Так как плитка вида $1 \times 4$ не содержит вовсе или содержит 2 клетки цвета 1. Значит замощение при замене плитки выполнить не получится.
\end{solution}

\chapter{Принцип Дирихле}

\begin{exercise}
В банде 50 бандитов. Все вместе они ни в одной разборке ни разу не участвовали, а каждые двое встречались на разборках ровно по разу. Докажите, что один из бандитов был не менее, чем на восьми разборках.
\end{exercise}

\begin{solution}
Предположим противное. Выберем произвольного бандита, полагая, что он был менее, чем на 8 разборках. С каждым из 49 оставшихся он должен встретиться не более раза, значит, на какой то разборке он должен был встретиться не менее чем с 7 бандитами. Всего на этой разборке было не менее 8 бандитов. Выберем бандита, который в этой разборке не принимал участие. Он искомый, так как с каждым из бандитов, бывших на этой разборке он должен был встретиться по разу. 
\end{solution}

\chapter{Теория графов}

\begin{exercise}
В графе 20 вершин, степень каждой не меньше 10. Доказать, что в нём есть гамильтонов путь.
\end{exercise}

\begin{solution}
Допустим противное. Тогда рассмотрим самый длинный из существующих путей, обозначив его $L$. Пусть в $L$ входит $n$ вершин. Вершины которые не входят в $L$ обозначим $M$. Вершины пути $L$ будем обозначать $L_{i}$ где $1 \leqslant i \leqslant n$. Так как степень $L_{n}$ не меньше 10 и это конечная вершина, то все ребра из нее должны идти в вершины $L$. Если из вершины $L_{i}$ есть ребро в вершину из $M$ и существует ребро из $L_{n}$ в $L_{i - 1}$ то мы можем сконструировать путь $L_{1} \dots L_{i-1}L_{n}\dots L_{i} M$. Длина этого пути будет больше $n$. Отсюда следует, что в $M$ могут идти только ребра тех вершин $L_{i}$, которые не соединены с $L_{n}$. Таких ребер не более $n - 10$. В компоненте $M$ не более $20 - n$ вершин, то есть степень вершины в $M$ не больше $19 - n$. Из $L$ в $M$ ведут не более $n - 10$ ребер. Значит максимальная степень вершины в $M$ равна $(19 - n) + (n - 10) = 9$. Противоречие.
\end{solution}

\begin{exercise}
Каждое из рёбер полного графа с 6 вершинами покрашено в один из двух цветов. Докажите, что есть три вершины, все рёбра между которыми - одного цвета.
\end{exercise}

\begin{solution}
Выберем произвольную вершину. Ее степень равна 5, значит, по принципу Дирихле, как минимум три ребра будут одного цвета. Пусть это будет синий цвет. Рассмотрим вершины к которым ведут эти ребра. Между ними не может быть синих ребер, так как тогда две вершины образовывали бы одноцветный треугольник с первой вершиной. Но в таком случае эти три вершины сами образовывают одноцветный треугольник. 
\end{solution}

\begin{exercise}
Каждое из рёбер полного графа с 9 вершинами покрашено в синий или красный цвет. Докажите, что либо есть четыре вершины, все рёбра между которыми - синие, либо есть три вершины, все рёбра между которыми - красные.
\end{exercise}

\begin{solution}
Выберем произвольные 6 вершин. Среди них должен быть треугольник все ребра которого одного цвета. Либо эти ребра красного цвета и тогда реализуется второй случай, либо все ребра синие. 
\end{solution}

\begin{exercise}
Каждое из рёбер полного графа с 17 вершинами покрашено в один из трёх цветов. Докажите, что есть три вершины, все рёбра между которыми - одного цвета.
\end{exercise}

\begin{solution}
Выберем произвольную вершину. Ее степень 16, значит, по принципу Дирихле, от нее отходят не менее 6 ребер одного цвета. Пусть этот цвет синий. Рассмотрим полный граф на 6 вершинах, к которым ведут эти ребра. Если в графе есть синее ребро, то найдутся три вершины соединенные ребрами синего цвета. Тогда допустим, что в этом подграфе нет синих ребер. Тогда он должен быть окрашен в два цвета, а в полном двуцветном графе на 6 вершинах найдется одноцветный треугольник.
\end{solution}

\begin{exercise}
Каждое из рёбер полного графа с 18 вершинами покрашено в один из двух цветов. Докажите, что есть четыре вершины, все рёбра между которыми - одного цвета.
\end{exercise}

\begin{solution}
Рассмотрим произвольную вершину. По принципу Дирихле, от нее отходят не менее 9 ребер одного цвета. Пусть этот цвет красный. В графе на 9 вершинах, к которым идут эти ребра, найдется три вершины, все ребра между которыми - красные или четыре вершины все ребра между которыми синие. В каждом случае мы получим клику на 4 вершинах, все ребра между которыми одного цвета.  
\end{solution}

\begin{exercise}
В правильном 21-угольнике шесть вершин покрашены в красный цвет, а семь вершин - в синий. Обязательно ли найдутся два равных треугольника, один из которых с красными вершинами, а другой - с синими?
\end{exercise}

\end{document} 
